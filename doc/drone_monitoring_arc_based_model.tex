\documentclass[preprint,review,12pt,3p,authoryear]{elsarticle}
% \documentclass[preprint,review,12pt,3p,authoryear]{elsarticle}

%% Use the option review to obtain double line spacing
%\documentclass[authoryear,preprint,review,12pt]{elsarticle}

%% Use the options 1p,twocolumn; 3p; 3p,twocolumn; 5p; or 5p,twocolumn
%% for a journal layout:
% \documentclass[final,1p,times]{elsarticle}
% \documentclass[final,1p,times,twocolumn]{elsarticle}
% \documentclass[final,3p,times]{elsarticle}
%% \documentclass[final,3p,times,twocolumn]{elsarticle}
% \documentclass[final,5p,times]{elsarticle}
%% \documentclass[final,5p,times,twocolumn]{elsarticle}

% \usepackage{cite}
\usepackage{amsmath,amssymb,amsfonts}
%\usepackage{algorithmic}
\usepackage{algorithm}
% \usepackage{enumitem}
\usepackage{enumerate}
\usepackage{mathtools}
\usepackage{graphicx,color}
\usepackage{textcomp}
% \usepackage{apacite}
\usepackage[noend]{algpseudocode}
%\usepackage{algpseudocode}
\usepackage[labelsep=period]{caption}
\usepackage{hyperref}
\usepackage{tabularx}
\usepackage{multirow}
\usepackage{diagbox}
\usepackage{booktabs}
\usepackage{natbib}
\usepackage{threeparttable}
\usepackage{algorithmicx}
% \usepackage[style=apa, backend=biber]{biblatex}
% \DeclareLanguageMapping{american}{american-apa}
% \addbibresource{ref.bib}

\newtheorem{proposition}{Proposition}
\newproof{proof}{Proof}
\newcolumntype{C}{>{\centering\arraybackslash}X}
\DeclareCaptionLabelFormat{figlabel}{Fig.~#2}
\captionsetup[figure]{labelformat=figlabel}

\renewcommand{\figureautorefname}{Fig.}
\algnewcommand{\LineComment}[1]{\State \(\triangleright\) #1}

\journal{Transportation Research Part E}

\begin{document}

\begin{frontmatter}

\title{Arc-Flow Model for Multi-Drone Persistent Monitoring Problem}

\author[inst1]{Author Name}
\ead{author@email.com}

\address[inst1]{Department/Institution, City, Country}

\begin{abstract}
This document presents an arc-flow mathematical model for the multi-drone persistent monitoring problem. The model is formulated on a time-expanded network and aims to minimize the total flight time of drones while ensuring continuous coverage of all target points. The model considers drone endurance constraints and flow conservation requirements.
\end{abstract}

\begin{keyword}
Drone routing \sep Persistent monitoring \sep Arc-flow model \sep Time-expanded network \sep Mathematical programming
\end{keyword}

\end{frontmatter}

\section{Mathematical Model}

This section presents the arc-flow formulation for the multi-drone persistent monitoring problem on a time-expanded network.

\subsection{Parameters}

\textbf{Parameters:}
\begin{itemize}
    \item $N$: the set of nodes in the time-expanded network, which includes the depot-time node set $N_0$ and the target-time node set $N_V$.
    \item $N_i$: The set of right-adjacent nodes to node $i$.
    \item $K$: The set of drones.
    \item $\delta_{vt}^i$: Whether (v,t) is observed by the drone at node $i$.
\end{itemize}

\subsection{Decision Variables}

\textbf{Decision Variables:}
\begin{itemize}
    \item $x_{ij}^k \in \{0,1\}$: Whether drone $k$ traverses the edge $(i,j)$. $i,j \in N$ (but the target points of $i$ and $j$ are not allowed to be the same), $k \in K$.
    \item $y_t^k \in \{0,1\}$: The number of drones $k$ available at time $t$ (whether it takes off). $t \in T, k \in K$.
    \item $q_i^k \in \{0,1,\ldots,T_d\}$: The remaining endurance time of drone $k$ at node $i$.
\end{itemize}

\subsection{Arc-Flow Formulation}

\textbf{Arc-Flow Model:}
\begin{align}
    \text{Min} \quad & \sum_{k \in K} \sum_{i \in N} \sum_{j \in N_i} (t_j - t_i) x_{ij}^k \tag{1} \\
    \text{s.t.} \quad & \sum_{j \in N_V} x_{ij}^k - \sum_{j \in N_V} x_{ji}^k \leq y_{t_i - 1}^k - y_{t_i}^k \quad \forall i \in N_0, k \in K \tag{2} \\
    & \sum_{j \in N_V} x_{ij}^k \leq 1 \quad \forall k \in K, i \in N_0 \tag{3} \\
    & \sum_{j \in N_V} x_{ji}^k \leq 1 \quad \forall k \in K, i \in N_0 \tag{4} \\
    & \sum_{j \in N} x_{ji}^k = \sum_{j \in N} x_{ij}^k \quad \forall k \in K, i \in N_V \tag{5} \\
    & T_d - (t_j - t_i) x_{ij}^k \geq q_j^k \quad \forall i \in N_0, j \in N_i, k \in K \tag{6} \\
    & q_i^k - (t_j - t_i) x_{ij}^k + T_d (1 - x_{ij}^k) \geq q_j^k \quad \forall i \in N_V, j \in N_i, k \in K \tag{7} \\
    & \sum_{k \in K} \sum_{j \in N_i} \sum_{\substack{i \in N_V | v_i = v}} x_{ij}^k \delta_{vt}^i \geq 1 \quad \forall t \in T, v \in V \tag{8} \\
    & x_{ij}^k \in \{0,1\} \tag{9} \\
    & y_t^k \in \{0,1\} \tag{10} \\
    & q_i^k \in \{0,1,\ldots,T_d\} \tag{11}
\end{align}

\subsubsection{Constraint Explanations}

\begin{itemize}
    \item \textbf{Constraint (1)}: Objective function that minimizes the total flight time of all drones.
    \item \textbf{Constraint (2)}: Flow balance constraint at depot nodes, relating the difference between outgoing and incoming flows to drone availability.
    
    \textit{Note on periodic scheduling:} To support periodic scheduling, time indices follow modular arithmetic. Specifically, when $t_i - 1 < 0$, we define $t_i - 1 = |T| - 1$, i.e., $t_i - 1 \equiv (t_i - 1) \mod |T|$. This allows the model to handle cyclic monitoring scenarios where the planning horizon repeats continuously.
    \item \textbf{Constraint (3)}: Each drone can depart from a depot node at most once.
    \item \textbf{Constraint (4)}: Each drone can return to a depot node at most once.
    \item \textbf{Constraint (5)}: Flow conservation constraint at target nodes - incoming flow equals outgoing flow.
    \item \textbf{Constraint (6)}: Initial endurance constraint when departing from the depot.
    \item \textbf{Constraint (7)}: Endurance propagation constraint when traveling between target nodes.
    \item \textbf{Constraint (8)}: Coverage constraint ensuring that each target $v$ is observed by at least one drone at each time period $t$.
    \item \textbf{Constraints (9-11)}: Domain definitions for decision variables.
\end{itemize}

\section{Model Characteristics}

The proposed arc-flow model has the following key features:

\begin{enumerate}
    \item \textbf{Time-Expanded Network}: The model operates on a time-expanded network where nodes represent location-time pairs, enabling explicit modeling of temporal aspects.
    
    \item \textbf{Persistent Monitoring}: Through constraint (8), the model ensures continuous coverage of all targets throughout the planning horizon.
    
    \item \textbf{Endurance Management}: Constraints (6) and (7) track the remaining endurance of each drone, ensuring that no drone exceeds its maximum flight time $T_d$.
    
    \item \textbf{Multi-Drone Coordination}: Multiple drones can cooperate to achieve complete coverage, with the model determining optimal routes and schedules.
    
    \item \textbf{Objective Optimization}: The model minimizes total flight time, which is relevant for energy efficiency and operational cost reduction.
\end{enumerate}

%% References (uncomment if needed)
% \bibliographystyle{elsarticle-harv}
% \bibliography{ref}

\end{document}

